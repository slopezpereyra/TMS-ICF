
\documentclass{article}
\usepackage{amsmath}
\usepackage{ragged2e}
\usepackage{amssymb}
\usepackage[utf8]{inputenc}
\usepackage{pmboxdraw}
\usepackage{graphicx} \graphicspath{
{./images/} }
\usepackage{indentfirst}
\usepackage{multicol}
\usepackage[a4paper, total={6.5in, 8.5in}]{geometry}
\DeclareSymbolFont{letters}{OML}{ztmcm}{m}{it}
\newtheorem{theorem}{Theorem}
\newtheorem{wrnote}{Writing note}
\newtheorem{proof}{Proof}

\begin{document} 

\begin{multicols}{2}
\section{Introduction}
\newtheorem{definition}{Definition}
\newtheorem{remark}{Remark}

Transcranial magnetic stimulation (TMS) is a common,
non-invasive experimental technique used to evoke action
potentials in cortical regions of the brain. In particular,
researchers often target the motor cortex and measure the
motor evoked potential (MEP) aroused by the stimulation.
When the purpose of the experiment is to inquire on
neuroplasticity, such stimulations are performed under the
\textit{paired pulse paradigm}.

The \textit{paired pulse paradigm} (or \textit{double pulse
paradigm}) consists in eliciting a series of two temporally
proximate pulses (in the order of miliseconds). The evoked
potentials of the double-stimulation are compared to those
of single test stimulations, and their relative amplitude is
taken as a proxy of neuroplasticity in the brain. The time
separating each of the paired pulses is termed
\textit{interstimulus interval} (ISI). It is the general
case that low intervals (4 or 5 miliseconds) produce
intracortical inhibition, with the evoked potentials of
paired stimulations being generally lower than those of
single pulses. Greater intervals, on the other hand, tend to
produce facilitation. 


In the context of this paper, we shall term any coefficient
that serves to represent the proportional relationship
between the potentials evoked by paired and test pulses a
\textit{measure of relative amplitude}. Measures of relative
amplitude in neuroscience are generally computed at the
subject level. This is reasonable, since hypotheses
generally deal with differences across subject groups.
However, TMS experimental results are pulse-specific evoked
potentials, and transforming them to subject- or
group-specific measures implies down-scaling the data resolution. 

The goal of this paper is to provide pulse-specific measures of relative
amplitude. This is, coefficients that represent the relative amplitude of each
individual paired pulse with respect to the set of test pulses in an
experimental session. The purpose of this endeavor is to keep data resolution
at its highest, which on its turn allows for the implementation of data-driven
artificial intelligence in the analysis of the 
experimental results. Thus, from a computer science perspective, our objective is
constrained to the sphere of feature engineering. 

We will show how pulse-specific measures of relative amplitude allow for
otherwise unfeasible computational analyses of TMS data, such as the use of
machine learning models for the detection of different pulse-response patterns
among different groups of clinical subjects. In particular, we will show they
allow for a machine learning classifier to correctly determine whether a
subject belongs to one of four clinical categories, based only on its evoked
potentials and across different inter-stimulus intervals, with an accuracy of up to
$90\%$.

\section{Relative amplitude features}

For simplicity, we will deal with the hypothetical situation
in which a single ISI was used for paired stimulations. All
of our results generalize to different inter-stimulus intervals.

\subsection{Definitions}

Let $k$ be the number of experimental subjects in some
subject group $\mathcal{G}$, to each of whom $n$ paired stimulations and
$m$ test stimulations were elicited.

\begin{definition}

Let $\textbf{P}^{n \times k}, \textbf{T}^{m\times k}$ be matrices representing
the paired and test potentials evoked across each of the $k$ subjects, such that  

\begin{equation*} \textbf{P} := \begin{bmatrix}x_{11} &
    x_{12}& \ldots & x_{1k} \\ x_{21} & x_{22}& \ldots &
    x_{2k} \\ \vdots & \vdots &\ddots& \vdots \\ x_{n1} & x_{n2}& \cdots & x_{nk}
\end{bmatrix} ~~ \textbf{T}:= \begin{bmatrix}t_{11} &
t_{12}& \ldots & t_{1k} \\ t_{21} & t_{22}& \cdots & t_{2k} \\
      \vdots & \vdots &\ddots& \vdots \\ t_{m1} & t_{m2}& \cdots & t_{mk}
\end{bmatrix} \end{equation*} \end{definition}


\subsection{The $\rho$ and $\delta$ features}

\begin{definition} 
    Let $x \in \textbf{P}_{\star i}$ be the MEP of a single paired stimulation
    elicited on the $i$th experimental subject, and $\textbf{t} =
    \textbf{T}_{\star i}$ the vector containing the MEPs of all test pulses
    elicited on that subject. Then we define two pulse-specific relative
    amplitude measures,

    \begin{equation} 
        \rho(x) := \frac{mx}{\sum_{i=1}^mt_j}
    \end{equation}

    \begin{equation} 
        \delta(x) := \frac{x}{m}\sum_{j=1}^m\frac{1}{t_j} 
    \end{equation}
\end{definition}

\begin{remark} 
    $\forall x: x \in \mathbb{R}^+:\delta(x) \geq \rho(x)$. (For a proof of this
    property, consult the appendix.) 
\end{remark}

Notice that $\rho(x)$ is the proportion between the potential
$x$, evoked by a paired stimulation, with respect to the
average potential of single test stimulations. On the other
hand, $\delta(x)$ is the average proportion of $x$ with respect
to each single test pulse.

Each feature improves the performance of a random forests model in a similar
degree, as will be shown later. This corroborates that both capture different
but complementary information ---as it is intended in their formulations. In
particular, $\rho$ can be a useful representation of the relative importance of
each double pulse in relation to the overall distribution of the test pulses in
the subject. It measures the deviation of the double pulse $x$ with respect to
the average test pulse. On the other hand, $\delta$ is a measure of the
proportionality of a double pulse with respect to different values in the
(certainly exponential) distribution of the test pulses in the subject.

Since the neural response to transcranial stimulations follows a distribution
$\beta$ very close to exponential (see \textbf{Statistics} section), we
experimented with the inclusion of the features above applied to the
logarithmically transformed MEPs. The inclusion of both $\rho(x), \rho(\ln(x))$
for every $x$ improves the performance of a random forest model significantly,
and the same occurs for $\delta$, as can be seen in the \textbf{Empirical
results} section.

It must be said that the $\rho$ function is not an entirely new contribution.
Traditionally, the group-level relative amplitude measure is conceived as the
average, across all subjects in a group, of the average paired response divided
by the average test response. This means that, if we let $S_i$ be the average
relative amplitude of the $i$th subject, then $S_i$ has been traditionally
defined as

\begin{equation*} 
    S_i=\frac{\frac{1}{n}\sum_{j=1}^n x_{ji}}{\frac{1}{m}\sum_{j=1}^m t_{ji}} = 
    \frac{\rho(x_{1i})+\cdots + \rho(x_{ni})}{n} 
\end{equation*}

as it is easy to see from decomposing the sum in the
numerator. In other words, the traditionally used measure of
relative amplitude, at the subject level, has always been
the average $\rho$ in a subject, albeit it was never defined explicitly.

\subsection{The weighted variants $\rho_w, \delta_w$}

As stated earlier, action potentials evoked by transcranial magnetic
stimulations follow a Gamma distribution that is very close to the exponential.
(see \textbf{Statistics} section). Experimentation has shown that the inclusion
of weighted variants of the features defined above greatly improves the
performance of a random forests model (see \textbf{Empirical results}).

The use of weighted instead of arithmetic averages may be useful in dealing with
the excessive influence of outliers or highly spread out points in the feature.
For example, the weight vectors may be computed using the MAD or the
inverse-variance of each $x$. 

In our \textbf{Empirical results}, we show the influence of including
inverse-variance variance weights to the model. But the general formulation of
these alternative features, for any desired weight vector, is

\begin{align} 
    \rho_w(x) &:= \frac{xm\sum_{j=1}^m w_j}{\sum_{j=1}^m t_j w_j} \\ 
    \delta_w(x) &:= \frac{\frac{x}{m}\sum_{j=1}^m\frac{w_j}{t_j}}{\sum_{j=1}^mw_j} 
\end{align}

where $\textbf{w}$ is some appropriate weight vector.

\section{Statistics}

\begin{wrnote}
    The description of the experimental process in this section is poor. Also I
    do not know the total number of subjects, experiments are still being
    conducted. Need input from the lab on these matters.
\end{wrnote}

We used data collected across $N = ?$ subjects at the \textit{Laboratory for the
Study of Slow-wave sleep activity}, University of Pennsylvania, with $H = ?$
healthy controls and $D = ?$ diagnosed with major depressive disorder (MDD).
Transcranial magnetic stimulation of the motor cortex was elicited to them after
a night of baseline sleep and after a night of slow-wave disruption (SWD) sleep.
Motor evoked potentials were measured via an electrode (?) placed on the
subjects' thumb (?). In the slow-wave disruption session, an auditory stimulus with
sufficient strength to interrupt the normal occurrence of slow-wave sleep, yet
not strong enough to wake the subjects, was elicited. This experimental setting
produces four distinct categories, two depending on the subject group and two on
the type of sleep session underwent, as shown in the table below.

~

\begin{tabular}{ |p{2cm}|p{2cm}|p{2cm}|  }
\hline
& Baseline & Slow-wave disruption \\
\hline
Healthy control & HC BL & HC SWD \\
\hline
Major depressive disorder & MDD BL & MDD SWD \\
\hline
\end{tabular}

~


In our statistical analyses, we have contemplated test and paired pulses
separately, since they are different type of stimulations. We have found the
peak-to-peak EMG to follow an exponential distribution in both test and paired
pulses. This is true when taking different inter-stimulus intervals in
consideration as well as when observing the distribution of distinctly spaced
paired pulses. 

Although distributions were always exponential, the $\beta$ parameter of said
distributions varied across subject groups and session types. 

~

    \begin{tabular}{ |p{2cm}|p{2cm}|p{2cm}|  }
    \hline
    &       Mean & Median \\
    \hline
    HC BL & 196.77 & 149.36 \\
    \hline
    HC SWD & 165.98 & 99.76 \\
    \hline \\ 
    MDD BL & 187.56 & 127.69\\ 
    \hline 
    MDD SWD & 247.57 & 151.07 \\
    \hline
    \end{tabular}

~ 

Each of the means in the above table correspond to the estimated $\beta$
parameter of the distribution of the paired pulses on each subject group.

\end{multicols}
\raggedright

~

\includegraphics[scale=0.14]{Paired-TestDist}
\raggedright
\includegraphics[scale=0.14]{P-dist-subjects-big}



\justifying
\begin{multicols}{2}

Each observation in the data we used was a specific experimental observation
resulting of an individual transcranial stimulation. The original features of the data
were: 

\begin{enumerate}
    \item An $EMG$ variable with the EMG peak-to-peak of each observation.
    \item A $Label$ categorical feature, which encoded the group of
        the subject of each observation, was the target variable. 
    \item An $ISI$ feature that econded the inter-stimulus interval of each
        pulse. A value of $-1$ indicated that the given pulse was a test pulse (no
        inter-stimulus interval).
\end{enumerate}

To these features, we included (separately and then together) the engineered
features $\rho$ and $\delta$, which were computed for each observation using the
Julia programming language. The code used to compute the features is publicly
available (at \ldots).

\section{Empirical results}

The objective was to evaluate whether the inclusion of the engineered,
pulse-specific measures of relative amplitude improved the performance of a
machine learning model, and in what degree. In order to do this, we set a
random forests model with the task of classifying every observation with its
appropriate label. In order words, the model was set out to infer, based on the
properties of each transcranial stimulation response, the group of the subject
upon which the stimulation was elicited, as well as the type of sleep session
after which it occurred.

We trained the model first on the data without pulse-specific
measures of relative amplitude (this is, with only the three features listed in
the \textbf{Statistics} section). We then trained the exact same model on the
data with the inclusion of $\delta$ but not $\rho$, and the inclusion of $\rho$
but not $\delta$. We then trained the same model with both engineered features,
and lastly with both features and also their weighted variantes, using
inverse-variance weights.


\subsection{Raw data}

When trained on the raw data, without the inclusion of pulse-specific relative
amplitude measures, the model's accuracy was of $\approx34.2 \%$. However, the
confusion of the model shows the errors are concentrated on the healthy control
categories, while categorization of diagnosed subjects was substantially better.
This implies major depressive subjects show statistically significant and
distinct patterns in their TMS responses in comparison to healthy controls, and
may be considered evidence for depression-induced differences in
neuroplasticity.

\end{multicols}

\centering
~

\includegraphics[scale=0.4]{cm-raw}


\justifying
\begin{multicols}{2}

However, it is important to note that the proxy for neuroplasticity in the
double pulse paradigm is precisely the relative amplitude of double pulses with
respect to test pulses. The raw data lacks any measure of relative amplitude.
Besides, regardless of the fact that the model above suggests a critical
difference in the response patterns between subject groups, its accuracy is
still poor.

\subsection{Engineered data}

With the inclusion of the $\rho$ feature, accuracy increases by a factor of
$\approx 2.1$ to $72.6\%$. The inclusion of the $\delta$ feature alone increased
it, in comparison to the raw data, had a more or less equivalent impact,
increasing accuracy to $73.4\%$.


\end{multicols}
\raggedright

~

\includegraphics[scale=0.24]{cm-rho}
\raggedright
\includegraphics[scale=0.24]{cm-delta}



\justifying
\begin{multicols}{2}

The model still shows a higher accuracy when classifying diagnosed subjects, but
the overall accuracy across all categories was significantly improved.

As stated earlier, the inclusion of the $\delta$ and $\rho$ features computed
over the logarithmically transformed EMG peak-to-peak improved the model's
accuracy. Concretely, accuracy was increased to $86\%$, with the following
confusion matrix.

\end{multicols}

~
\centering

\includegraphics[scale=0.4]{cm-dr-ln}


\justifying
\begin{multicols}{2}

At last, if to the previous model we add also the weighted versions of $\delta$
and $\rho$, using inverse-variance weights, we obtained an accuracy of $89.8\%$,
with the following confusion matrix.

\end{multicols}

\centering
~

\includegraphics[scale=0.4]{final-cm}


\justifying
\begin{multicols}{2}

\section{Discussion}

The previous results show that the inclusion of pulse-specific relative
amplitude measures greatly improve the accuracy of a random forests model
trained over TMS data for subject classification. More importantly, it becomes
clear that, via the engineered features $\delta$ and $\rho$, researchers can
attain evidence in favour or against hypotheses pertaining to differences among
subject groups. Particularly, in showing that pulse-specific measures of
relative amplitude are significant information to determine the sleep session
and group of a subject, and being relative amplitude measures a proxy for
neuroplasticity, the previous model provides evidence in favour of
sleep-modulated, depression-induced differences in neuroplasticity.

The use of machine learning models over neuroscientific observations is not only
a promising tool in the production of evidence. There is a diagnostic potential
that is still to be evaluated. Indeed, if machine learning models can detect the
neural patterns that distinguish, under specific experimental conditions,
healthy control from diagnosed subjects, such models can potentially be
implemented in the diagnosis process as powerful clinical tool. 

Although long and serious scientific effort is still required to appraise the
diagnostic potential of machine learning models, we believe our results allow
for a certain amount of conservative optimism on the matter.

In short, pulse-specific relative amplitude features are successful in making
machine learning models applicable to TMS experimental data. Thus, they can play
an important part in future research by allowing for new ways of analyzing and
extracting meaningful information of TMS results.

\section{Appendix}

\textbf{Proof 1}. In \textbf{Remark 1}, we observed the following property:

    \begin{align*}
        \forall x: x \in \mathbb{R}^+ : \delta(x) \geq \rho(x)
    .\end{align*}

Such property can be proven via induction. Firstly, recall that


\begin{align*} \delta(x) &= \frac{x}{m}\sum^m\frac{1}{t_j} \\
                \rho(x) &= \frac{xm}{\sum^m t_j} \end{align*}


Let $S_1^m = \sum^m\frac{1}{t_j}, S_2^m= \sum^m t_j$. We operate under the
assumption that $t_i \in \mathbb{R}^+$. It is the case that

\begin{align*} 
    \frac{x}{m}\sum^m\frac{1}{t_j} &\geq \frac{xm}{\sum^m t_j} \\ 
    S_2^mS_1^m &\geq m^2
\end{align*}

This holds for $m=1$, since $\frac{1}{t_1}+t_1 \geq 1 \iff 1+t_1^2 \geq t_1$. So
we may assume $S^k_1 S^k_2 \geq k^2$ . We now set out to show that

\begin{equation*} 
    S^{k+1}_1 S^{k+1}_2 \geq (k+1)^2
\end{equation*}

This can be proven as follows.

\begin{align*} 
    S^{k+1}_1 S^{k+1}_2 &\geq (k+1)^2 \\ 
    (S_1^k+\frac{1}{t_{t+1}})(S_2^k+t_{k+1}) &\geq k^2+2k+1 \\
    S^k_1S^k_2+ t_{k+1}S_1^k + \frac{1}{t_{k+1}}S^k_2+1 &\geq k^2+2k+1 \\
    S^k_1S^k_2+ t_{k+1}S_1^k + \frac{1}{t_{k+1}}S^k_2 &\geq k^2+2k 
\end{align*}

We know $S^k_1S^k_2 \geq k^2$ and then it suffices to show $t_{k+1}S_1^k +
\frac{S^k_2}{t_{k+1}}\geq 2k$. To prove this, simply observe that

\begin{align*}
    \frac{1}{t_{k+1}}\sum_{j=1}^mt_j+t_{k+1}\sum_{j=1}^m\frac{1}{t_{j}} &\geq 2k \\
    \Big(\frac{t_1}{t_{k+1}}+...+\frac{t_k}{t_{k+1}}\Big)+\Big(\frac{t_{k+1}}{t_1}+...+\frac{t_{k+1}}{t_k}\Big)
    &\geq 2k \\ \iff
    \overbrace{a+\frac{1}{a}+b+\frac{1}{b}+... +
    n+\frac{1}{n}}^{\text{$2k$ terms} } &\geq 2k
\end{align*}

We have $\min f=2$  for $f(x)=x+\frac{1}{x}$ for $x \in \mathbb{R}^+$. Then
$\min(a+\frac{1}{a}+...+n+\frac{1}{n})=2k$ for $a,..., n \in \mathbb{R}^+$,
which concludes the demonstration.

\end{document}

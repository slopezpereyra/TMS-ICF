\documentclass{article}
\usepackage{amsmath}
\usepackage{amssymb}
\usepackage[utf8]{inputenc}
\usepackage{pmboxdraw}
\usepackage{graphicx} \graphicspath{
{./images/} }
\usepackage{indentfirst}
\usepackage{multicol}
\usepackage[a4paper, total={6.5in, 8.5in}]{geometry}
\DeclareSymbolFont{letters}{OML}{ztmcm}{m}{it}

\begin{document} 

\begin{multicols}{2}
\section{Introduction}
\newtheorem{definition}{Definition}
\newtheorem{remark}{Remark}

Transcranial magnetic stimulation (TMS) is a common,
non-invasive experimental technique used to evoke action
potentials in cortical regions of the brain. In particular,
researchers often target the motor cortex and measure the
motor evoked potential (MEP) aroused by the stimulation.
When the purpose of the experiment is to inquire on
neuroplasticity, such stimulations are performed under a
\textit{paired-pulse paradigm}.

The \textit{paired-pulse paradigm} (or \textit{double pulse
paradigm}) consists in eliciting a series of two temporally
proximate pulses (in the order of miliseconds). The evoked
potentials of the double-stimulations are compared to those
of single test stimulations, and their relative amplitude is
taken as a proxy of neuroplasticity in the brain. The time
separating each of the paired-pulses is termed
\textit{interstimulous interval} (ISI). It is the general
case that low intervals (4 or 5 miliseconds) produce
intracortical inhibition, with the evoked potentials of
paired stimulations being generally lower than those of
single pulses. Greater intervals, on the other hand, tend to
produce facilitation. 


In the context of this paper, we shall term any coefficient
that serves to represent the proportional relationship
between the potentials evoked by paired and test pulses a
\textit{measure of relative amplitude}. Measures of relative
amplitude in neuroscience are generally computed at the
subject level. This is reasonable, since hypotheses
generally deal with differences across subject groups.
However, TMS experimental results are pulse-specific evoked
potentials, and transforming them to subject- or
group-specific measures implies down-scaling the data resolution. 

The goal of this paper is to provide pulse-specific measures of relative
amplitude. This is, coefficients that represent the relative amplitude of each
individual paired-pulse with respect to the set of test pulses in an
experimental session. The purpose of this endeavor is to keep data resolution
at its highest, which on its turn allows for the implementation of data-driven
artificial intelligence in the analysis of the 
experimental results. Thus, from a computational perspective, our objective is
constrained to the sphere of feature engineering. 

We shall show how pulse-specific measures of relative amplitude allow for
otherwise unfeasible computational analyses of TMS data, such as the use of
machine learning models for the detection of different pulse-response patterns
among different groups of clinical subjects. In particular, we will show they
allow for a machine learning classifier to correctly determine whether a
subject belongs to one of four clinical categories based only on its evoked
potentials across different inter-stimulous intervals with an accuracy of up to
$90\%$.

\section{Relative amplitude features}

For simplicity, we will deal with the hypothetical situation
in which a single ISI was used for paired stimulations. All
of our results generalize to different ISI.

\subsection{Definitions}

Let $k$ be the number of experimental subjects in some
subject group, to each of whom $n$ paired stimulations and
$m$ test stimulations were elicited.

\begin{definition}

    Let $\textbf{P}^{n \times k}, \textbf{T}^{m\times k}$ be
    matrices representing the paired and test potentials
    evoked across each of the $k$ subjects, such that  

    \begin{equation} \textbf{P} := \begin{bmatrix}x_{11} &
    x_{12}& \cdots & x_{1k} \\ x_{21} & x_{22}& \cdots &
x_{2k} \\ & & ... & \\ x_{n1} & x_{n2}& \cdots & x_{nk}
\end{bmatrix} ~~ \textbf{T}:= \begin{bmatrix}t_{11} &
t_{12}& ... & t_{1k} \\ t_{21} & t_{22}& \cdots & t_{2k} \\
      & & ... & \\ t_{m1} & t_{m2}& \cdots & t_{mk}
\end{bmatrix} \end{equation} \end{definition}

        \begin{definition} Let $\mu_t := \begin{bmatrix} \mu_1 &
        \mu_2 & \ldots & \mu_k \end{bmatrix}^\top$ be such that 
$\mu_i$ is the average test potential for the $i$th subject.
In other words,

\begin{equation} \mu_i = \frac{1}{m}\sum_{j=1}^m t_{ji}
\end{equation} \end{definition}

\begin{remark} Evoked potentials of both paired and test
    pulses conform to a Gamma distribution, as discussed in
    the appendix (?). This means it is always an implicit assumption that
    $\forall (x \in \textbf{P}, t\in \textbf{T})|x, t
\in \mathbb{R}^+$. \end{remark}

\subsection{The $\rho$ and $\delta$ features}


\begin{definition} Let $x \in \textbf{P}_{\star i}$ be a
    single paired-stimulation MEP in the $i$th experimental
    subject, and $\textbf{t} = \textbf{T}_{\star i}$ the
    vector containing all test MEPs of that subject. Then we
    define two pulse-specific relative amplitude measures,

    \begin{equation} \rho(x) := \frac{mx}{\sum_{i=1}^mt_j}
    \end{equation}

    \begin{equation} \delta(x) :=
    \frac{x}{m}\sum_{j=1}^m\frac{1}{t_j} \end{equation}
\end{definition}

\begin{remark} $\forall x \in \mathbb{R}^+|\delta(x) \geq
\rho(x)$. (For a proof of this property, consult the
appendix.) \end{remark}

Notice that $\rho$ is the proportion between the potential
$x$, evoked by a paired stimulation, with respect to the
average potential of single test stimulations. On the other
hand, $\delta$ is the average proportion of $x$ with respect
to each single test pulse.


The $\rho$ function is not an entirely new contribution. The
traditional group level relative amplitude measure is
conceived as the average, across all subjects in a group, of
the average paired response divided by the average test
response of each subject. If we let $S_i$ be the average
relative amplitude of the $i$th subject, then $S_i$ has ben
traditionally defined as

\begin{equation} S_i=\frac{\frac{1}{n}\sum_{j=1}^n
x_{ji}}{\frac{1}{m}\sum_{j=1}^m t_{ji}} =
\frac{\rho(x_{1i})+\cdots + \rho(x_{ni})}{n} \end{equation}

as it is easy to see from decomposing the sum in the
numerator. In other words, the traditionally used measure of
relative amplitude, at the subject level, has always been
the average $\rho$ in a subject. 

\subsection{The weighted variants $\rho_w, \delta_w$}

As stated earlier, action potentials evoked by trasncranial
magnetic stimulation follow a Gamma distribution; one that
closely resembles an exponential distribution (see
\textbf{Statistics} section). A valid form of data
augmentation that also attempts to deal with the presence of
outliers is to weight the averages involved in $\rho,
\delta$ using inverse-variance weights. The attempt is to
allow for potentials that lay at the tail of the
distribution to contribute in a way proportional to their
probability. 

If we let $\rho_w, \delta_w$ be the weighted versions of
$\rho, \delta$, then we have 

        \begin{align} \rho_w(x) &:= \frac{xm\sum_{j=1}^m
        w_j}{\sum_{j=1}^m t_j w_j} \\ \delta_w(x) &:=
    \frac{\frac{x}{m}\sum_{j=1}^m\frac{w_j}{t_j}}{\sum_{j=1}^m
w_j} \end{align}

where $\textbf{w}$ is some appropriate weight vector. Since
the purpose of $\textbf{w}$ is to reduce the contribution of
outliers to the overall measure of relative amplitude, it is
to be expected that the distribution of $\textbf{w}$ closely
resembles that of evoked potentials.

\section{Empirical results}

We used data collected across $N = ?$ subjects at the
\textit{Laboratory for the Study of Slow-wave sleep activity},
University of Pennsylvania, with $H = ?$ healthy controls
and $D = ?$ diagnosed with major depressive disorder (MDD).
Transcranial magnetic stimulation of the motor cortex was
elicited to them after a night of baseline sleep and after a
night of slow-wave disruption (SWD) sleep. Motor evoked
potentials were measured via an electrode (?) placed on the
subjects' thumb. In the slow-wave disruption session, an
auditory stimulus with sufficient strength to interrupt the
normal occurrence of slow-wave sleep, yet not strong enough
to wake the subjects, was elicited. This experimental
setting produces four distinct categories, two depending on
the subject group and two on the type of sleep session
underwent.

We trained the exact same random forests model first on the
data without pulse-specific measures of relative amplitude,
then on the data including some or all the pulse-specific
measures of relative amplitude. The model's task was to
classifiy each TMS stimulation into one of the four
experimental categories. In other words, it faced the
challenge of establishing to which type of subject, and
under which type of session, each stimulation was elicited,
based only on the evoked potential, the ISI and ---later---
the relative amplitude measures of the stimulation. Our
objective was to evaluate whether the inclusion of
pulse-specific measures of relative amplitude improved the
model's accuracy, and if so in what degree. 


\subsection{Raw data}

When trained on the raw data, without the inclusion of
pulse-specific relative amplitude measures, the model's
accuracy was $A\approx34.2 \%$.

\pagebreak
\section{Appendix}

\subsection*{Proofs}

\textbf{Theorem: } $\forall x \in \mathbb{R}^+|\delta(x)
\geq \rho(x)$

Recall that

                \begin{align} \delta(x) &=
                \frac{x}{m}\sum^m\frac{1}{t_j} \\ \rho(x) &=
            \frac{xm}{\sum^m t_j} \end{align}


            Let $S_1^m = \sum^m\frac{1}{t_j}, S_2^m= \sum^m
            t_j$. We operate under the assumption that $t_i
            \in \mathbb{R}^+$. It is the case that

                    \begin{align}
                    \frac{x}{m}\sum^m\frac{1}{t_j} &\geq
                \frac{xm}{\sum^m t_j} \\ \iff S_2^mS_1^m
                                                   &\geq m^2
                \end{align}

                This holds for $m=1$, since
                $\frac{1}{t_1}+t_1 \geq 1 \iff 1+t_1^2 \geq
                t_1$. So we may assume $S^k_1 S^k_2 \geq
                k^2$ . We may only show now that



                \begin{equation} S^{k+1}_1 S^{k+1}_2 \geq
                (k+1)^2 \end{equation}

                    \begin{align} S^{k+1}_1 S^{k+1}_2 &\geq
                        (k+1)^2 \\
                        (S_1^k+\frac{1}{t_{t+1}})(S_2^k+t_{k+1})
                        &\geq k^2+2k+1 \\ S^k_1S^k_2+
                        t_{k+1}S_1^k +
                        \frac{1}{t_{k+1}}S^k_2+1 &\geq
                        k^2+2k+1 \\ S^k_1S^k_2+ t_{k+1}S_1^k
                    + \frac{1}{t_{k+1}}S^k_2 &\geq k^2+2k
                \end{align}

                We know $S^k_1S^k_2 \geq k^2$ and then it
                suffices to show $t_{k+1}S_1^k +
                \frac{S^k_2}{t_{k+1}}\geq 2k$. To prove
                this, simply observe that

                        \begin{align}\frac{1}{t_{k+1}}\sum_{j=1}^mt_j+t_{k+1}\sum_{j=1}^m\frac{1}{t_{j}}
                        &\geq 2k \\
                        \iff\frac{1}{t_{k+1}}(t_1+t_2+...+t_k)+t_{k+1}(\frac{1}{t_1}+\frac{1}{t_2}+...+\frac{1}{t_k})
                        &\geq 2k \\ \iff
                        \Big(\frac{t_1}{t_{k+1}}+\frac{t_2}{t_{k+1}}+...+\frac{t_k}{t_{k+1}}\Big)+\Big(\frac{t_{k+1}}{t_1}+\frac{t_{k+1}}{t_2}+...+\frac{t_{k+1}}{t_k}\Big)
                        &\geq 2k \\ \iff
                        \overbrace{a+\frac{1}{a}+b+\frac{1}{b}+...
                    + n+\frac{1}{n}}^{\text{$2k$ terms} }
                        &\geq 2k \end{align}

                        We have $\min f=2$  for
                        $f(x)=x+\frac{1}{x}$ for $x \in
                        \mathbb{R}^+$. Then
                        $\min(a+\frac{1}{a}+...+n+\frac{1}{n})=2k$
                        for $a,..., n \in \mathbb{R}^+$,
                        which concludes the demonstration.



                        \end{document}
